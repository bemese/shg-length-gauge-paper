\section{Introduction}\label{intro}

%check past tense
%nonlocal, not non-local

Nonlinear optics has been a relevant topic in physics research for the past 
fifty years and has gained a wide range of applications in 
diverse areas of technology and research. Second-harmonic generation (SHG) is
one of the most fundamental nonlinear processes, and is of particular 
importance in material science where it is has been proven to be very 
effective as a surface probe. \cite{hughesPRB96, guyot-sionnestPRB88, 
lupkeSSR99, downerPSSA01, shenAPB99, shenNAT89, chenPRL81, 
maytorenaPRB98} Surface spectroscopy is a crucial tool for research in 
microelectronics, semiconductors, nanomaterials, and many more recent 
topics of great scientific and commercial interest.

Numerous theoretical models have been developed to study the nonlinear optical
response of materials -- each with various disadvantages. We recognize two 
formalisms in this context. The first, called the ``velocity gauge'' or 
``$\mathbf{v}\cdot\mathbf{A}$ gauge,'' features in early research in the field. 
Butcher and McLean,\cite{butcherPOPS63}, and Aspnes\cite{aspnesPRB72} laid the 
groundwork for the theory. They derived expressions for the nonlinear 
susceptibility tensor that were either diverged at low frequencies or were 
limited to specific crystal classes. Two decades later, 
Ghahramani \emph{et al}.\cite{ghahramaniPRB91} realized a full-band-structure 
calculation for Si/Ge superlattices and found significant surface SHG 
while providing further proof that the expressions do not necessarily diverge 
at low frequencies for certain materials. Sipe \emph{et al}.\cite{sipePRB93}
distinguish both interband and intraband contributions to the susceptibility 
tensor and further verify the non-divergent behavior. Levine\cite{levinePRB94} 
introduces the scissors operator within the LDA and finds good agreement 
with experimental results. The second formalism, called the ``length gauge'' 
or ``$\mathbf{r}\cdot\mathbf{E}$ gauge'' presents various advantages over 
the velocity gauge. Aversa and Sipe\cite{aversaPRB95} presented divergence 
free expressions for the susceptibility tensor using the length gauge that 
are  significantly easier to derive and calculate. Cabellos \emph{et al}.
\cite{cabellosPRB09} provide a detailed derivation of the length gauge 
formalism and a comparison with the velocity gauge for bulk GaAs. They 
prove that the results are indeed gauge-invariant. This presents a strong
case for using the length over the velocity gauge in order to more 
efficiently calculate the nonlinear response of a system.

We are also concerned with the optical response of surfaces and interfaces.
Mendoza and Moch\'an\cite{mendozaPRB96} produced an early review of 
theoretical SHG spectra from surfaces. Arzate and Mendoza\cite{arzatePRB01}
went into a detailed analysis of the different contributions to the SHG spectra
of a surface and the nuanced relationship between bulk, surface, interband, 
intraband, $1\omega$ and $2\omega$ terms. Calculated surface SHG spectra are 
in good agreement with experiment using both the velocity gauge
\cite{mendozaPRB01,mendozaPRL98,reiningPRB94} and the length-gauge 
approach\cite{sharmaPRB03}. Perhaps of more relevance to this 
work, Mendoza \emph{et al}.\cite{mendozaPRB06} and 
Mejía \emph{et al}.\cite{mejiaRMF04} developed a layer-by-layer analysis of 
the linear and nonlinear responses of semiconductor systems. This model
allows for obtaining results from selected regions within a slab or bulk,
including the surface.

The inclusion of nonlocal effects, such as the scissors correction, has been
correctly introduced in both the velocity and length guage schemes. Both 
Rashkeev \emph{et al}.,\cite{rashkeevPRB98} and Hughes and 
Sipe\cite{hughesPRB96} effectively introduce the scissors correction when 
calculating the nonlinear susceptibility tensor. Leitsmann 
\emph{et al}.\cite{leitsmannPRB05} includes excitonic and local field 
interactions within the velocity gauge scheme. 
Lastly, Nastos \emph{et al}.\cite{nastosPRB05} demonstrated the correct 
way to calculate the nonlinear response using the scissored Hamiltonian 
providing better agreement with experiment. 

The most complete approach will be a combination of the different techniques 
that lead to more efficient and accurate calculations. Therefore,
the goal of this paper is to derive new expressions for the nonlinear
susceptibility tensor $\chi^{abc}(-2\omega;\omega,\omega)$ in the length
gauge formalism, that includes nonlocal terms from the scissors approximation 
and the nonlocal spatial contribution from the pseudopotentials, and a 
layer-by-layer approach convenient for surface SHG calculations. We are also
able to divide the resulting tensor in interband, intraband, $1\omega$, and 
$2\omega$ contributions. This approach is unprecedented and yet produces 
results that are in total agreement with the literature.

The paper is organized as follows. In Sec. \ref{theory} we present the 
relevant steps for deriving the surface second-order susceptibility tensor 
within the length gauge formalism. This derivation includes the addition of 
nonlocal terms and a layer-by-layer analysis that have been omitted 
from previous works. In Sec. \ref{results} we show the results of this 
reformulation with a layer-by-layer study of a clean Si(111) surface with a 
$2 \times 1$ reconstruction. These calculations are done with ab initio 
programs based on density-functional theory (DFT) within the local-density 
approximation (LDA), using Troullier-Martins pseudopotentials. We compare 
results from before and after adding the different nonlocal contributions.
Finally, in Sec. \ref{conclusions} we present our conclusions.
